\chapter{Opis techniczny}
Głównym tematem tego rozdziału jest opis techniczny wszystkich aplikacji i skryptów, które 
tworzą całe narzędzie. Została tu opisana struktura kodu i schemat komunikacji między 
aplikacjami.
Kod projektu można podzielić na cztery główne części:
\begin{enumerate}
  \item interfejs użytkownika,
  \item serwer komunikujący się z zewnętrznymi API oraz bazą danych,
  \item kod służący do instalacji zależności i uruchamiania aplikacji na maszynie wirtualnej 
  - korzystający z technologii Vagrant i Ansible,
  \item skrypt w języku Python konwertujący wstawki w języku \LaTeX do obrazów.
\end{enumerate}

\section{Serwer}
Kod serwera i inne pliki potrzebne do jego działania znajdują się w katalogu \texttt{forms-app}.

\subsection{Kod źródłowy}
Kod źródłowy serwera znajduje się w katalogu \texttt{src}. Serwer udostępnia kilka adresów,
pod którymi można się z nim komunikować używając protokołu http:
\begin{itemize}
  \item \texttt{GET /forms}: zwraca wszystkie formularze, które znajdują się w lokalnej
    bazie danych,
  \item \texttt{GET /forms/:id}: zwraca informacje o formularzu o podanym identyfikatorze,
  \item \texttt{POST /forms}: pod ten adres można wysłać formularz w kodowaniu JSON, żeby
    stworzyć nowy formularz Google,
  \item \texttt{PUT /forms/:id}: pod ten adres można wysłać zakodowany formularz w formacie
    JSON, żeby edytować formularz o podanym identyfikatorze, edytowany jest formularz
    w lokalnej bazie danych oraz formularz Google,
  \item \texttt{GET /forms/:id/answers}: zwraca wszystkie przesłane odpowiedzi,
  \item \texttt{GET /forms/:id/scores}: zwraca wszystkie ocenione przesłane odpowiedzi,
  \item \texttt{GET /forms/:id/scores}: przesyła plik EXCEL z ocenionymi przesłanymi 
    odpowiedziami,
  \item \texttt{DELETE /forms/:id}: usuwa formularz o podanym identyfikatorze z lokalnej bazy
    danych oraz usuwa wszystkie pytania z formularza Google, ale nie jego samego
\end{itemize}
Kod z konfiguracją tych adresów można znaleźć w pliku \texttt{src/server.js}. Aplikacja
z interfejsem użytkownika używa tych adresów, żeby komunikować się z serwerem.
Plik \texttt{formsFunctions.js} zawiera funkcje pomocnicze, które przykładowo tworzą
zapytania wysyłane do Google Forms API lub oceniają zamknięte pytania w przesłanych
odpowiedziach i konwertują je do formatów JSON lub EXCEL. Plik \texttt{jsonValidator.js}
zawiera walidator zakodowanych formularzy w formacie JSON, który jest dokładnie opisany
w pracy \ap.

\subsection{Google Forms API}
W poprzedniej wersji projekt korzystał z Google Apps Script. Część kodu oraz utworzone
formularze Google były przechowywane na jednym konkretnym koncie Google, którego ewentualna
zmiana byłaby bardzo skomplikowana. Było to niewygodne w użyciu, może niezbyt bezpieczne
oraz utrudniało rozwój narzędzia. Dlatego zdecydowałam się przepisać część komunikującą się
serwerami Google na kod wysyłający zapytania do Google Forms API. Google Forms API to niedawno
powstałe API, które w łatwy sposób pozwala zarządzać formularzami Google na dowolnym koncie
Google i jest wciąż dynamiczne rozwijane, więc w przyszłości można liczyć na dodanie do niego
wielu funkcjonalności, które pozwoliłoby na dalszy rozwój narzędzia, które jest tematem tej pracy.
Żeby korzystać z Google Forms API wystarczy założyć projekt w konsoli Google Cloud, co jest w pełni
darmowe dla każdej osoby posiadającej konto Google i jest dokładnie opisane w następnym rozdziale.
Następnie użytkownik może pozwolić aplikacji zarządzać formularzami na swoim dysku Google bez obaw, 
że ktokolwiek inny miałby do nich dostęp. Serwer wysyła do Google Forms API zapytania i dostaje
przejrzyste odpowiedzi w formacie JSON. Na ten moment API pozwala zarządzać formularzami oraz
przesłanymi do nich odpowiedziami, co pozwala na napisanie naszego własnego sposobu oceniania tych
odpowiedzi. Google Forms API udostępnia automatyczne ocenianie, ale za każde pytanie można
dostać tylko zero lub maksymalną liczbę punktów, co jest niewystarczające. Chcielibyśmy móc
za każde pytanie przydzielić przykładowo połowę punktów, jeśli nie zostały zaznaczone wszystkie
poprawne odpowiedzi. W tej pracy udało się to zrealizować.

\subsection{Imgur API}
% nowy template odpowiedzi

\subsection{Lokalna baza danych}
 
\subsection{Katalog assets}
Wszystkie obrazki z tekstem ze wstawkami w języku \LaTeX  oraz wszystkie inne pliki, które
powstają podczas ich tworzenia są generowane w katalogu \texttt{assets}, dlatego ważne jest, żeby
go nie usuwać. Zawartość tego katalogu można usunąć oprócz pliku \texttt{.gitignore}, ponieważ
wygenerowane obrazki są używane tylko podczas tworzenia lub edytowania formularza.

\subsection{Katalog src/credentials}

\subsection{Katalog src/excels}

\subsection{ESLint}
Do aplikacji serwera został dodany ESLint, który pomaga pisać przejrzysty kod i na bieżąco
analizuje kod i sprawdza, czy nie zawiera on błędów. To zachowanie można dowolnie edytować używając 
zasad udostępnionych przez to narzędzie. Zasady skonfigurowane dla tego projektu można znaleźć 
w pliku \texttt{.eslintrc.json}. W celu dokładniejszego zrozumienia, jak działa to narzędzie 
zachęcam do zapoznania się z dokumentacją.