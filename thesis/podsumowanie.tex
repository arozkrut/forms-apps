\chapter{Podsumowanie}
Celem pracy był rozwój narzędzia służącego do zarządzania formularzami \mbox{Google}
i pomagającego
przeprowadzać testy oraz ankiety, używając tych formualarzy. Udało się dopisać kilka 
nowych funkcjonalności
i ulepszyć te już istniejące, ale wciąż istnieje wiele ulepszeń, które można dodaći funkcjonalności,
które można ulepszyć. W dalszej częsci podsumowania opiszę kilka z nich.

\subsection*{Automatyczne rozpoczynanie testów w podanym terminie}
W tej wersji narzędzia użytkownik musi ręcznie utworzyć formularz lub go zaktualizować dokładnie 
w tym momencie, w którym chciałby, żeby inni użytkownicy mogli zobaczyć dany formularz Google 
i wysyłać do niego odpowiedzi. Z pewnością wygodne byłoby dodanie możliwości podania daty, kiedy 
formularz ma zostać utworzony lub zaktualizowany.

\subsection*{Automatyczne kończenie testów w podanym terminie}
Podobnie wygodne byłoby kończenie testów w podanym terminie. Użytkownik musi ręcznie
kliknąć, że nie chce akceptować odpowiedzi w ustawieniach formularza na stronie Google,
kiedy chciałby zakończyć test, ale w przszłości, kiedy Google Forms API doda taką możliwość,
można to zaprogramować.

\subsection*{Zapamiętywanie poprzednich wersji formularzy}

Jeśli użytkownik chce automatycznie oceniać przesłane
odpowiedzi, nie może on edytować danego formularza po rozpoczęciu testu, ponieważ nawet pytania, 
które już istnieją
dostałyby nowe identyfikatory w bazie danych i niemożliwa byłaby identyfikacja pytań w poprzednio 
przesłanych odpowiedziach. Ten problem można rozwiązać poprzez zapisywanie w bazie danych poprzednich
wersji formularza razem z datami edycji. Baza danych Google pamięta wszystkie przesłane odpowiedzi,
nawet te odpowiadające na już usunięte pytania, więc wystarczyłaby zmiana modelu w lokalnej bazie 
danych. Wtedy można by też po zakończeniu testu usunąć pytania 
i tylko wyświetlać informację, że test się już zakończył i automatyczne ocenianie wciąż by działało.

\subsection*{Automatyczne włączenie opcji losowej kolejności pytań i zbierania adresów email}
Google Forms API wciąż nie oferuje wielu przydatnych funkcji takich jak włączenie losowej kolejności 
pytań w formularzu i zbierania adresów email. Jednak wciąż jest bardzo dynamicznie rozwijane, więc
możliwe, że taka opcja pojawi się w przyszłości. Można by wtedy dodać te dwie rzeczy do narzędzia.

\subsection*{Usuwanie formularzy z dysku Google}
Na ten moment po usunięciu formularza w narzędziu, plik z formularzem Google nie jest usuwany
z dysku Google użytkownika. Jest to spowodowane tym, że Google Forms API nie udostępnia takiej opcji,
ale za to Google Drive API już to robi. Trzeba by więc dodać Google Drive API do projektu i
zaimplementować usuwanie podanego pliku.