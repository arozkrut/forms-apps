\chapter{Środowisko}
Praca jest rozszerzeniem innego projektu, wobec tego część użytych bibliotek zostało opisanych 
w rozdziale ,,Środowisko'' pracy \ap. W niniejszym rozdziale skupię się więc na nowo dodanych
elementach środowiska. W celu zapoznania się z całością polecam lekturę wymienionej 
pracy.
\\ Kod aplikacji można podzielić na trzy główne części:
\begin{enumerate}
  \item interfejs użytkownika,
  \item serwer komunikujący się z zewnętrznymi API oraz bazą danych,
  \item kod służący do instalacji zależności i uruchamiania aplikacji na maszynie wirtualnej 
  - korzystający z technologii Vagrant i Ansible.
\end{enumerate}
Dodatkowo aplikacja korzysta z Google Forms API oraz Imgur API.

\section{Interfejs użytkownika}

\subsection{TypeScript}
Aplikacja została napisana w języku TypeScript, który jest nadzbiorem języka JavaScript. 
Aplikacja korzysta z bibliotek: 
types/node, typescript.

\subsection{React}
Technologia służąca do tworzenia interfejsów
graficznych aplikacji internetowych. Aplikacja korzysta z bibliotek:
emotion/react, emotion/styled, types/react, types/react-dom, 
react, react-dom, react-scripts.

\subsection{Material UI}
Biblioteka, która udostępnia wiele gotowych komponentów, których można użyć w
aplikacjach korzystających z Reacta. 

\subsection{axios}
Biblioteka służąca do wysyłania zapytań http i pobierania danych z serwera.

\subsection{date-fns}
Biblioteka pomagająca zarządzać datami. Udostępnia wiele użytecznych funkcji.

\subsection{react-code-blocks}
Biblioteka z gotowymi komponentami służącymi do wyświetlania bloków kodu.

\section{Serwer}

\subsection{Node.js}
Serwer został napisany w środowisku Node.js, które służy do tworzenia aplikacji
serwerowych w języku JavaScript.

\subsection{Google}
Aplikacja używa bibliotek google-cloud/local-auth oraz googleapis/forms
odpowiednio do autoryzacji projektu oraz użytkownika w serwerach Google 
oraz do wysyłania zapytań do Google Forms API, które jest opisane w dalszej sekcji.

\subsection{excel4node}
Biblioteka służąca do tworzenia plików Excel.

\subsection{lowdb}
Mała, lokalna baza danych przechowująca dane w plikach JSON. Za jej pomocą
przechowywane są zakodowane formularze.

\section{Vagrant i Ansible}
Vagrant to narzędzie służące do tworzenia wirtualnych środowisk programistycznych
z użyciem na przykład VirtualBox, ale współdziała też z wieloma innymi oprogramowaniami.
Ansible to narzędzie służące do automatyzacji
wdrażania, konfiguracji i zarządzania. Z pomocą tych dwóch technologi zostały
napisane skrypty, które same skonfigurują maszynę wirtualną i następnie uruchomią
na niej wszystkie aplikacje potrzebne do działania narzędzia. 

\section{Google Forms API oraz Imgur API}

\subsection*{Google Forms API}
Google Forms API to API, które umożliwia zarządzania Formularzami Google na dysku
użytkownika. W czasie pisania tej pracy jest to wciąż bardzo nowe narzędzie, ponieważ
jego pierwsza oficjalna wersja została opublikowana w 2022 roku. API pozwala m.in. zarządzać
pytaniami i odpowiedziami, zmieniać ustawienia formularza i włączać automatyczne ocenianie.
Jest ono wciąż dynamicznie rozwijane.

\subsection*{Imgur API}
Imgur API pozwala zarządzać plikami użytkownika na platformie Imgur, w tym dodawać je jako
zalogowany lub niezalogowany użytkownik. Obrazy z pytaniami, które zostają utworzone, jeśli
pytanie zawiera wstawki w języku \LaTeX, zostają wgrane na serwery Imgur, żeby serwery Google
mogły je pobrać używając zewnętrznego URL podczas tworzenia formularza.

\section{Python}
Narzędzia korzysta ze skryptu w języku Python, aby stworzyć obrazy ze wstawkami w języku \LaTeX.
Jest on dokładnie opisany we wspomnianej wcześniej pracy. Zostało dodane do niego wgrywanie 
wygenerowanych obrazków na serwery Imgur za pomocą biblioteki imgurpython.
Wszystkie biblioteki z jakich korzysta skrypt to: imgurpython, tex2pix, pdf2image oraz
opencv-python.

\section{Poppler}
Poppler to biblioteka służąca do renderowania plików PDF. Wspomniany powyżej skrypt
używa pakietu poppler-utils, do tworzenia plików PDF. Ten pakiet jest powszechnie
używany na systemach operacyjnych opierających się na Debianie.

\section{LaTeX}
Jak już wiele razy zostało wspomniane w tej pracy, narzędzie pozwala generować formularze ze 
wstawkami
w języku \LaTeX, zatem aby narzędzie działało poprawnie \LaTeX musi być zainstalowany 
na maszynie, na którym narzędzie ma zostać uruchomione. Skrypt Ansible dodatkowo instaluje
polski pakiet językowy.
