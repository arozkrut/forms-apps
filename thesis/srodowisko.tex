\chapter{Środowisko}
Praca jest rozszerzeniem innego projektu, wobec tego część użytych bibliotek zostało opisanych 
w rozdziale ,,Środowisko'' w pracy \ap. Nie chcę powtarzać opisu elementów zaimplementowanych 
w bazowej pracy, więc w tym rozdziale skupię się na nowo dodanych
elementach środowiska.
\\ Kod projektu można podzielić na cztery główne części:
\begin{enumerate}
  \item interfejs użytkownika,
  \item serwer komunikujący się z zewnętrznymi API oraz bazą danych,
  \item kod służący do instalacji zależności i uruchamiania aplikacji na maszynie wirtualnej 
  - korzystający z technologii Vagrant i Ansible
  \item skrypt w języku Python konwertujący wstawki w języku \LaTeX\ do obrazów.
\end{enumerate}
Dodatkowo aplikacja korzysta z Google Forms API oraz Imgur API.

\section{Interfejs użytkownika}

\subsection{TypeScript}
Aplikacja została napisana w języku TypeScript\cite{TypeScript},
który jest nadzbiorem języka JavaScript. 
Wszystkie programy napisane w TypeScripcie tłumaczą się bezpośrednio do języka JavaScript
oraz wszystkie programy napisane w języku JavaScript są poprawnymi programami w języku TypeScript.
TypeScript wnosi kilka nowych możliwości np. pozwala na opcjonalne statyczne typowanie 
i programowanie obiektowe oparte na klasach. 
Aplikacja korzysta z bibliotek: Definitely Typed\cite{DefinitelyTyped}, typescript\cite{typescript_}.

\subsection{React}
React\cite{React} to technologia służąca do tworzenia interfejsów
graficznych aplikacji internetowych. Jest szybki oraz oferuje wiele możliwości,
które ułatwiają implementowanie aplikacji internetowych, głównie tzw. Single Page Applications.
Przykładowo React przechowuje w pamięci wirtualny DOM aplikacji i po zmianie jego stanu porównuje
wirtualny i prawdziwy DOM, po czym aktualizuje jego stan w przeglądarce. Sprawia to, że strona
aplikacji zmieni się tylko, gdy faktycznie zostały wprowadzone jakieś zmiany
i zmieniają się tylko niektóre elementy,
a nie cała strona. Dzięki temu aplikacje korzystające z Reacta są szybkie.
Oprócz tego React używa języka JSX. Jest on nakładką na język JavaScript, która umożliwia
wstawianie kodu HTML lub komponentów React bezpośrednio w kodzie. 
Aplikacja korzysta z bibliotek:
emotion\cite{emotion}, react\cite{react_}, Create React App\cite{CreateReactApp}.

\subsection{Material UI}
Zazwyczaj nie ma sensu implementować od początku komponentów,
które są często używane w niemal każdej aplikacji internetowej, ponieważ istnieje już
wiele gotowych rozwiązań. Biblioteka Material UI\cite{MaterialUI} udostępniająca wiele
gotowych komponentów, których można użyć w aplikacjach korzystających z Reacta. 
Te komponenty są bardzo proste w użyciu i
można łatwo zmienić ich wygląd według własnych preferencji.
Dodatkowo Material UI posiada obszerną i przejrzystą dokumentację.

\subsection{axios}
Biblioteka axios\cite{axios} służy do wysyłania zapytań http. Bazuje ona na typie Promise z języka
JavaScript. Można jej użyć zarówno w aplikacjach używających Node.js, jak i
aplikacjach przeglądarkowych. W projekcie służy do wysyłania zapytań do serwera,
które zwracają dane o formularzach lub ocenach, tworzą nowe formularze lub 
edytują już istniejące.

\subsection{date-fns}
Biblioteka date-fns\cite{date-fns} to lekka biblioteka pomagająca zarządzać datami.
Udostępnia wiele użytecznych funkcji, takich jak formatowanie dat do 
ładnego napisu w tym w języku polskim, obliczenia na datach (dodawanie dat, zmiana strefy czasowej,
porównywanie dat, obliczanie okresu pomiędzy wydarzeniami itp.), edytowanie ich i wiele 
innych - w zasadzie wszystko, co może być przydatne w operowaniu datami. 
Można jej używać w języku TypeScript. 

\subsection{react-code-blocks}
Biblioteka react-code-blocks\cite{react-code-blocks} zawiera gotowe komponenty 
służące do wyświetlania bloków kodu. Pozwalają one kolorować składnię w zależności
od języka, zmieniać motywy i dowolnie zmieniać ich wygląd w łatwy sposób. Bibliotekę
można dodać do projektów korzystających z Reacta.  

\section{Serwer}

\subsection{Node.js}
Serwer został napisany w środowisku Node.js\cite{Node.js},
które służy do tworzenia aplikacji
serwerowych w języku JavaScript. Jest to narzędzie działające asynchronicznie, 
które rozpoczyna wykonywanie odpowiednich funkcji w reakcji na jakieś wydarzenia. 
Stanowi klasyczny przykład środowiska wykorzystującego język JavaScript. 

\subsection{Google}
Aplikacja używa bibliotek Google Local Auth: Node.js Client\cite{local-auth-node}
oraz Google APIs Node.js Client\cite{googleapis-forms}
odpowiednio do autoryzacji projektu oraz użytkownika w serwerach Google 
oraz do wysyłania zapytań do Google Forms API. Google APIs Node.js Client 
pomaga budować zapytania oraz zawiera wiele typów wykorzystywanych przez
Google Forms API.

\subsection{excel4node}
Biblioteka excel4node\cite{excelfornode} służy do tworzenia plików Excel.
Została stworzona do współpracy ze środowiskiem Node.js. Udostępnia funkcje
i klasy, które umożliwiają m.in wygenerowanie pliku, dodawanie zmiennych różnych
typów np. napisów, liczb i dat do odpowiednich komórek, przeprowadzanie obliczeń na
podanych komórkach oraz stylizowanie komórek np. zmienianie koloru i rozmiaru
czcionki, koloru komórek i wiele innych.

\subsection{lowdb}
lowdb\cite{lowdb} to mała, lokalna baza danych przechowująca dane w plikach
w formacie JSON.
Została napisane w czystym JavaScripcie i można z niej korzystać w środowisku 
Node.js. Pozwala ona w łatwy sposób przeprowadzać podstawowe operacje CRUD.
Można z niej korzystać synchronicznie lub asynchronicznie.
Za jej pomocą przechowywane są zakodowane formularze. 

\section{npm}
Narzędzie npm\cite{npm} to menadżer pakietów, który jest używany w aplikacji serwera oraz interfejsu użytkownika.
npm zarządza pakietami wykorzystywanymi przez te aplikacje. Listy zależności aplikacji, 
wersje tych zależności
oraz skrypty, w tym skrypt, który uruchamia aplikację interfejsu użytkownika można znaleźć
w pliku \texttt{package.json}.

\section{Vagrant i Ansible}
Vagrant\cite{vagrant} to narzędzie służące do tworzenia wirtualnych środowisk 
programistycznych z użyciem na przykład VirtualBox, ale współdziała też z wieloma
innymi oprogramowaniami.
Ansible\cite{ansible} to narzędzie służące do automatyzacji
wdrażania, konfiguracji i zarządzania. Z pomocą tych dwóch technologi zostały
napisane skrypty, które same skonfigurują maszynę wirtualną i następnie uruchomią
na niej wszystkie aplikacje potrzebne do działania narzędzia. 

\section{Google Forms API oraz Imgur API}

\subsection*{Google Forms API}
Google Forms API\cite{google-forms-api} to interfejs, który umożliwia zarządzania Formularzami Google na dysku
użytkownika. W czasie pisania tej pracy jest to wciąż bardzo nowe narzędzie, ponieważ
jego pierwsza oficjalna wersja została opublikowana w 2022 roku. API pozwala m.in. zarządzać
pytaniami i odpowiedziami, zmieniać ustawienia formularza i włączać automatyczne ocenianie.
Jest ono wciąż dynamicznie rozwijane.

\subsection*{Imgur API}
Imgur API\cite{imgur-api} pozwala zarządzać plikami użytkownika na platformie Imgur, w tym dodawać je jako
zalogowany lub niezalogowany użytkownik. Jeśli pytanie, które ma zostać dodane do formularza, 
zawiera wstawki w języku \LaTeX\, generowane są obrazy z tekstem pytania. Następnie 
zostają one wgrane na serwery Imgur, aby potem serwery Google
mogły je pobrać używając zewnętrznego URL.

\section{Python}
Narzędzie korzysta ze skryptu w języku Python\cite{python}, aby stworzyć obrazy ze wstawkami w języku \LaTeX.
Jest on dokładnie opisany we wspomnianej wcześniej pracy. Zostało dodane do niego wgrywanie 
wygenerowanych obrazków na serwery Imgur za pomocą biblioteki imgurpython.
Wszystkie biblioteki z jakich korzysta skrypt to: imgurpython\cite{imgurpython},
tex2pix\cite{tex2pix}, pdf2image\cite{pdf2image} oraz
opencv-python\cite{opencv-python}.

\section{Poppler}
Poppler\cite{poppler} to biblioteka służąca do renderowania plików PDF. Wspomniany powyżej skrypt
używa pakietu poppler-utils, do tworzenia plików PDF. Ten pakiet jest powszechnie
używany na systemach operacyjnych opierających się na Debianie.

\section{LaTeX}
Jak już wiele razy zostało wspomniane w tej pracy, narzędzie pozwala generować formularze ze 
wstawkami
w języku \LaTeX\cite{latex}, zatem aby narzędzie działało poprawnie \LaTeX\ musi być zainstalowany 
na maszynie, na której narzędzie ma zostać uruchomione. Skrypt Ansible dodatkowo instaluje
polski pakiet językowy.\\
\LaTeX{} jest  językiem programowania służącym do tworzenia plików \*.pdf. Dodatkowo \LaTeX{}
umożliwia wstawianie do tekstu symboliki matematycznej, tabel, diagramów, schematów
oraz ułatwia utrzymanie estetycznego i czytelnego układu tekstu.

