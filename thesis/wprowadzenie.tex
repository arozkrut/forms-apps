\chapter{Wprowadzenie}%Wstęp - opis problemu, motywacja

Wraz z rozpoczęciem pandemii problem zdalnego sprawdzania wiedzy i umiejętności
 zrobił się popularny. Miejsca, w których było to szczególnie uciążliwe to m.~in. szkoły, 
uczelnie, czy procesy rekrutacyjne w różnych firmach. Dobre narzędzie do przeprowadzania zdalnych
testów powinno spełniać pewne wymagania, takie jak:
\begin{itemize}
  \item przejrzystość formularzy,
  \item łatwość tworzenia nowych testów,
  \item możliwość skutecznego weryfikowania tożsamości osoby piszącej test.
\end{itemize}
Dodatkowo dobrze jest, gdy takie narzędzie posiada również takie funkcje jak:
\begin{itemize}
  \item automatyczne sprawdzanie pytań zamkniętych,
  \item kontrolowanie czasu potrzebnego poszczególnym osobom na zakończenie testu,
  \item możliwość losowych zmian kolejności pytań,
  \item możliwość zapisywania odpowiedzi w popularnym formacie.
\end{itemize}
Ponadto niektórzy egzaminujący cenią sobie możliwość wstawiania symboli matematycznych, 
wykresów czy zdjęć w ramach pytań.

 We wrześniu 2021 roku Agnieszka Pawicka obroniła pracę inżynierską o tytule \ap, 
 w której zaproponowała system umożliwiający generowanie testów Google Form z pliku 
 w formacie JSON. Projekt umożliwiał także generowanie pytań zawierających wstawki w \LaTeX, 
 które były automatycznie kompilowane i wstawiane do formularza.
 \\ Celem tej pracy jest usprawnienie działania wymienionego wyżej narzędzia.
 \section{Wprowadzone usprawnienia}
W pracy został zaimplementowany szereg modyfikacji, które ułatwiają korzystanie z narzędzia
zarówno przy tworzeniu formularzy, zarządzaniu nimi, jak i ich późniejszym sprawdzaniu. Te usprawnienia to:
\begin{itemize}
  \item łatwiejszy sposób instalacji narzędzia,
  \item zmiana interfejsu użytkownika na bardziej intuicyjny,
  \item zmiana sposobu wgrywania zakodowanego formularza,
  \item możliwość edycji pliku JSON z zakodowanym formularzem już po wgraniu go,
  \item możliwość zobaczenia odpowiedzi oraz wyników w formacie JSON,
  \item możliwość pobrania danych o przesłanych odpowiedziach 
   i wynikach sprawdzania automatycznego w formacie EXCEL,
  \item utworzone formularze są teraz przechowywane na Dysku Google użytkownika 
  - nie jak w poprzedniej wersji na wspólnym dysku dla wszystkich użytkowników,
  \item szereg nowych opcji związanych z parametrami formularza tj.:
  \begin{itemize}
    \item możliwość zmiany kolejności odpowiedzi wewnątrz pytań zamknietych,
    \item możliwość dodania  opisu testu,
    \item odpowiedzi do pytań mogą być teraz w formie symboli  matematycznych (\LaTeX),
    \item możliwe jest dodanie oceniania innego niż zerojedynkowe dla pytań
     wielokrotnego wyboru.
  \end{itemize}  
\end{itemize}

Wszystkie wymienione zmiany wraz z użytymi narzędziami, instrukcją użytkownika oraz opisem
środowsk i napotkanych problemów są opisane w dalszych rozdziałach niniejszej pracy. 

